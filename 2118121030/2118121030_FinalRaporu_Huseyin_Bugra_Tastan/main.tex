\documentclass{article}
\usepackage[utf8]{inputenc}
\usepackage{titling}
\usepackage{pgfgantt}
\usepackage{graphicx}
\usepackage{pdflscape}
\renewcommand{\refname}{Kaynakça}
\renewcommand{\figurename}{Resim}
\pretitle{
  \begin{center}
  \LARGE\bfseries
  \includegraphics[width=0.2\textwidth]{ksbu.png} % Başlık önüne ekleyeceğiniz bir logo
  \vskip 1em
}
\title{Reinforcement Learning ile Yapay Zekaya Hayatta Kalmayı Öğretme}
\author{Hüseyin Buğra Taştan}
\date{13 Haziran 2024}

\begin{document}

    \maketitle
    \begin{center}
        \includegraphics[width=0.4\textwidth]{forest.jpg} 
    \end{center}
    \vfill
    \rule{\textwidth}{0.5pt}
    \renewcommand{\abstractname}{Özet}
\begin{abstract}
\noindent "AI Learns To Survive" adlı proje, derin güçlendirme öğrenme (deep reinforcement learning) kullanarak yapay zeka ajanlarının hayatta kalma becerilerini öğrenmeye odaklanan bir araştırma veya uygulama projesidir. Projede amaç, yapay zeka ajanlarını belirli bir ortamda hayatta kalmaya yönlendirmek ve bu süreçte karşılaştıkları zorluklarla başa çıkabilmelerini sağlamaktır.
\end{abstract}
\rule{\textwidth}{0.5pt}
    \vfill

\newpage
\section{Giriş}
\rule{\textwidth}{0.5pt}
Günümüzde, yapay zeka teknolojisinin hızla gelişmesiyle birlikte, yapay zeka ajanlarının gerçek dünya ortamlarında başarılı bir şekilde işlev görmesi ve hatta hayatta kalabilmesi giderek daha büyük bir önem kazanıyor. "AI Learns To Survive" (Yapay Zeka Hayatta Kalmayı Öğreniyor) projesi, bu bağlamda derin güçlendirme öğrenme tekniklerini kullanarak yapay zeka ajanlarının hayatta kalma becerilerini geliştirmeye odaklanan heyecan verici bir araştırma ve uygulama girişimidir.Projemizin temel amacı, yapay zeka ajanlarını belirli bir ortamda hayatta kalmaya teşvik etmek ve bu süreçte karşılaşacakları çeşitli zorluklarla başa çıkabilmelerini sağlamaktır. Bu zorluklar, gerçek dünya senaryolarından esinlenmiş olabileceği gibi sanal ortamlarda da simüle edilebilir.Örneğin, bir robotun doğal afetler, engeller veya diğer tehlikelerle karşılaştığı bir ortamda hayatta kalabilme yeteneği üzerine odaklanabiliriz.Derin güçlendirme öğrenme algoritmaları, projemizin temelini oluşturur. Bu algoritmalar, yapay zeka ajanlarının ortama uyum sağlamak ve belirlenmiş hedefleri başarmak için optimal eylemleri öğrenmelerini sağlar. Öğrenme süreci genellikle bir ödül sistemiyle desteklenir; ajanlar belirli görevleri başarıyla tamamladıklarında veya belirli zorlukları aştıklarında ödüllendirilirler. Bu ödül sistemi, ajanların istenilen davranışları öğrenmelerini teşvik eder."Aİ Learns To Survive" projesi, yapay zeka alanında önemli bir boşluğu doldurmayı hedefliyor. Gerçek dünya uygulamalarında kullanılabilen yapay zeka ajanlarının, değişen ve bazen de tehlikeli ortamlarda başarılı bir şekilde işlev görebilmesi, teknolojinin ilerlemesinde kritik bir adımdır. Bu proje, bu hedefe ulaşmada bir adım daha atmaktadır.
  \newpage
  \section{Literatür Çalışması}
\rule{\textwidth}{0.5pt}
Bu makalede, derin pekiştirmeli öğrenme yöntemlerini kullanarak, yüksek boyutlu duyusal girdilerden (örneğin, görüntü veya konuşma gibi) doğrudan ajanları kontrol etmeyi öğrenmektir. Peşin başarılı pekiştirmeli öğrenme uygulamaları, bu tür alanlarda genellikle elle hazırlanmış özelliklerle birleştirilmiş doğrusal değer fonksiyonları veya politika temsillerine dayanmıştır\cite{mnih2013playing}.\\[15pt]

Bu makalede,derin pekiştirmeli öğrenme modellerini daha hızlı ve daha verimli bir şekilde eğitmek ve genişletmek için asenkron yöntemlerin etkin bir şekilde kullanılmasıdır. Bu sayede, daha karmaşık ve gerçekçi problemler üzerinde daha iyi performans gösteren yapay zeka sistemleri geliştirmek mümkün olabilir\cite{mnih2016asynchronous}.\\[15pt]

Bu projede, derin sinir ağları ve ağaç araması gibi ileri öğrenme tekniklerinin kullanılmasıyla, Go oyununu oynamak için daha etkili bir yaklaşım geliştirilmiştir. Derin sinir ağları, oyun tahtasındaki durumu analiz etmek ve olası hamleleri tahmin etmek için kullanılırken, ağaç araması algoritması, bu tahminleri daha ileri seviyeye taşımak ve olası hamlelerin sonuçlarını değerlendirmek için kullanılır\cite{silver2016mastering}.\\[15pt]

Bu proje, derin takviyeli öğrenme tekniklerini kullanarak genel video oyunları için yapay zeka geliştirmeyi amaçlamaktadır. Derin takviyeli öğrenme, yapay zeka ajanlarının belirli bir görevi gerçekleştirmek için çevreleriyle etkileşimde bulunarak deneyimlerinden öğrenmelerini sağlayan bir makine öğrenimi yaklaşımıdır.\cite{torrado2018deep}


\newpage


\clearpage


\section{ML Agents Nedir?}
ML Agents, Unity'nin yapay zeka ve makine öğrenimi modellerini eğitmek için geliştirilmiş bir araç setidir. Bu araç seti, Unity oyun motorunda oyun nesnelerini kontrol etmek için kullanılan yapay zeka algoritmalarını ve modellerini eğitmek için bir dizi araç ve kütüphane sunar. ML Agents, karmaşık oyun senaryolarında yapay zeka davranışlarını modellendirme ve eğitme sürecini kolaylaştırır.

\subsection{ML Agents Kullanımı}
ML Agents kullanarak yapay zeka modelleri eğitmek ve entegre etmek oldukça basittir. İşte temel adımlar:

\subsubsection{Çevrenin Tanımlanması}
İlk adım, yapay zeka modelini eğitmek için bir çevre tanımlamaktır. Bu çevre, oyun sahnesindeki nesneleri, hedefleri ve etkileşimleri içerir.

\subsubsection{Eğitim Verilerinin Toplanması}
Sonraki adım, çevrede yapay zeka modelini eğitmek için kullanılacak verilerin toplanmasıdır. Bu genellikle insan ya da önceden belirlenmiş stratejilerle oluşturulmuş verileri içerir.

\subsubsection{Yapay Zeka Modelinin Eğitimi}
Toplanan veriler kullanılarak yapay zeka modeli eğitilir. ML Agents, çeşitli makine öğrenimi tekniklerini destekler ve eğitim sürecini kolaylaştırır.

\subsubsection{Yapay Zeka'nın Entegrasyonu}
Eğitilen yapay zeka modeli, Unity oyun motoruna entegre edilir ve gerçek zamanlı olarak oyun sahnesinde kullanılır.
\clearpage
\section{Kurulum Aşamaları}
\begin{enumerate}
    \item Anaconda Kurulumu
    \item Unity Kurulumu
    \item ML-Agents Kurulumu
    \item Sanal Orrtam Oluşumu
    \item Gerekli Eklenti Kurulumları
    \item Eğitimin Başlatılması
\end{enumerate}
 \section{Proje Taslağı}
 Hayatta kalma oyunları genellikle oyuncuların doğal kaynakları toplamalarını, inşa etmelerini ve düşmanlardan kaçınmalarını veya onlarla savaşmalarını gerektirir. Yapay zeka, bu tür oyunlarda oyuncuların karşılaşabileceği düşman davranışlarını, kaynak toplama stratejilerini ve oyun dünyasıyla etkileşimlerini simüle etmek için kullanılabilir.
\begin{figure}[h]
    \centering
    \includegraphics[angle=90,width=0.9\textwidth]{taslak.jpeg}
    \caption{Proje Taslağı}
    \label{fig:resim7}
\end{figure}
\clearpage
\section{Ortam Hazırlama Adımları}

\subsection{Temel Oyun Dünyası Oluşturma}

\begin{itemize}
\item \textbf{Proje Oluşturma:} Unity'de yeni bir proje oluşturun ve gerekli ayarları yapın.
\item \textbf{Sahne Oluşturma:} Sahneye bir arazi veya ortam ekleyin. Bu, oyunun geçeceği dünyayı temsil eder. Toprak, ağaçlar, kayalar gibi doğal unsurları içerebilir.
\item \textbf{Işıklandırma:} Işıklandırmayı ayarlayın. Gündüz ve gece döngüsü gibi dinamik ışıklandırma efektleri eklemek, oyun dünyasını daha gerçekçi hale getirebilir.
\item \textbf{Detaylar ve Atmosfer:} Ortama detaylar ekleyin; bitkiler, taşlar, su vb. Bunlar oyunun atmosferini zenginleştirecek ve oyuncuların etkileşimini artıracaktır.
\end{itemize}

\begin{figure}[h]
    \centering
    \includegraphics[width=0.9\textwidth]{unityortam.PNG}
    \caption{Ortam Oluşturma}
    \label{fig:resim8}
\end{figure}

\subsection{Yapay Zeka İçin Temel Unsurlar}
\begin{itemize}
\item \textbf{Düşman Karakterler:} Düşman karakterlerinizi tasarlayın ve oluşturun. Bu karakterler, oyuncuya karşı saldırabilir veya onlardan kaçabilir.
\item \textbf{Davranış Kodlaması:} Düşmanların hareket ve davranışlarını belirleyen temel kodları yazın veya hazır yapay zeka çözümlerini kullanın. Örneğin, düşmanların oyuncuyu takip etmesi veya belirli bir alanı savunması gerekebilir.
\item \textbf{Düşman AI Entegrasyonu:} Oyun içinde düşman yapay zekasını etkinleştirecek ve onların oyuncularla etkileşimini sağlayacak kodları entegre edin.
\end{itemize}

\subsection{Kaynak Toplama ve İnşa Etme Sistemleri}

\begin{itemize}
\item \textbf{Kaynakları Tanımlama:} Oyuncunun kaynakları toplayabileceği ve bunları kullanarak yapılar inşa edebileceği bir sistem oluşturun. Bu kaynaklar odun, taş, besin gibi şeyleri içerebilir.
\item \textbf{Yapıları Tanımlama:} Oyuncunun inşa edebileceği yapıları belirleyin. Bu yapılar barınaklar, savunma kuleleri, üretim tesisleri vb. olabilir.
\item \textbf{İnşa ve Üretim Mekanikleri:} Oyuncunun kaynaklarını kullanarak yapılar inşa etmesini ve bu yapılar aracılığıyla kaynakları işlemesini sağlayacak mekanikleri tasarlayın ve kodlayın.
\end{itemize}

\begin{figure}[h]
    \centering
    \includegraphics[width=0.9\textwidth]{kaynaklar.PNG}
    \caption{Kaynak Belirleme}
    \label{fig:resim9}
\end{figure}

\subsection{Yapay Zeka ile Etkileşimler}

\begin{itemize}
\item \textbf{Düşman Etkileşimleri:} Yapay zekanın kaynakları toplaması ve inşa etmesi için gerekli kodları yazın veya entegre edin. Ayrıca, düşmanların oyuncunun inşa ettiği yapıları hedef alması veya onları yok etmeye çalışması için gerekli kodları ekleyin.
\item \textbf{Oyuncu-Yapı Etkileşimleri:} Oyuncunun inşa ettiği yapılarla yapay zeka arasında etkileşimleri sağlayın. Örneğin, düşmanların oyuncunun inşa ettiği yapıları hedef alması veya bu yapıları savunması gerekebilir.
\end{itemize}

\section{Trigger Kullanımı}

Trigger'lar, bir nesnenin içine giren diğer bir nesne tarafından algılanan alanlardır. Bir nesne bir Trigger alanına girdiğinde, bu olay bir tetikleyici (trigger) olarak adlandırılır ve belirli bir işlevi başlatır. Örneğin, bir oyuncunun bir tarlaya girdiğinde bir tetikleyici tetiklenerek hasat yapabilir.

\begin{figure}[h]
    \centering
    \includegraphics[width=0.9\textwidth]{collider.PNG}
    \caption{Trigger ve Collider Bileşeni}
    \label{fig:resim10}
\end{figure}

\subsection{Trigger'ı Etkinleştirme}

Unity'de bir nesnenin bir Trigger olarak işlev görmesi için, bir Collider bileşenine sahip olması ve "Is Trigger" özelliğinin etkinleştirilmiş olması gerekir.

\begin{figure}[h]
    \centering
    \includegraphics[width=0.9\textwidth]{triggeretkınlesme.PNG}
    \caption{Trigger Etkinleştirme}
    \label{fig:resim11}
\end{figure}
\clearpage
\subsection{Trigger Olayları}

Bir Trigger ile ilişkili olaylar genellikle OnTriggerStay, OnTriggerEnter ve OnTriggerExit fonksiyonlarıyla ele alınır. OnTriggerEnter, bir nesne Trigger alanına girdiğinde tetiklenir. OnTriggerStay, bir nesne Trigger alanında kaldığı sürece devam eder. Son olarak, OnTriggerExit, bir nesne Trigger alanından çıktığında tetiklenir.

\section{Collider Kullanımı}

Collider, nesnelerin fiziksel çarpışmalarını algılamak için kullanılır. İki nesne çarpıştığında, Collider'lar bu çarpışmayı algılar ve Unity'nin fizik motoruna bilgi gönderir, böylece nesneler arasındaki etkileşim gerçekleşir.

\subsection{Collider Türleri}

Unity'de farklı Collider türleri vardır, bunlar arasında BoxCollider, SphereCollider, CapsuleCollider, ve MeshCollider bulunur. Her biri farklı şekil ve formlara sahiptir ve nesnelerin çarpışma algılama şekillerini etkiler.
\subsection{Fiziksel Özellikler}

Collider'lar, nesnelerin fiziksel özelliklerini belirlemek için kullanılabilir. Örneğin, bir nesnenin ağırlığını, sürtünmesini ve elastikiyetini ayarlamak için Collider bileşeninin özellikleri kullanılabilir.

\begin{figure}[h]
    \centering
    \includegraphics[width=0.9\textwidth]{rigidbody.PNG}
    \caption{Rigidbody}
    \label{fig:resim12}
\end{figure}
\clearpage
\section{Örnek Kod Parçası}

\begin{verbatim}
using UnityEngine;

public class yemek_hasat : MonoBehaviour
{
    
    private void OnTriggerEnter(Collider other)
    {
        Envanter.foodCount += 1;
        
    }
}
\end{verbatim}

\begin{verbatim}
using UnityEngine;
using TMPro;

public class Envanter : MonoBehaviour
{ 
    public static int foodCount = 0;
    
    [SerializeField] private TextMeshProUGUI foodText;

    // Güncelleme kare başına bir kez çağrılır
    void Update()
    {
        foodText.text = foodCount.ToString();

    }
}
\end{verbatim}

Yukarıdaki örnek, çiftliğin bir oyuncu tarafından hasat edilmesini sağlamak için bir Trigger kullanır. OnTriggerEnter fonksiyonu, oyuncu bir Trigger alanına girdiğinde tetiklenir ve foodCounta 1 ekler.
\clearpage

\section{Pekiştirmeli Öğrenme Algoritması}
\begin{figure}[h]
    \centering
    \includegraphics[width=0.8\textwidth]{ajantablosu.png}
    \caption{Makine Öğrenimi \cite{mnih013playing}} 
    \label{fig:resim13}
\end{figure}
\begin{enumerate}
    \item \textbf{Gözlemleme:} Model, çevresindeki ortamı gözler ve bu ortam hakkında bilgi toplar. Örneğin, bir video oyunundaki bir karakter, oyunun dünyasını ve karakterin durumunu gözlemleyebilir.
    \item \textbf{Karar Verme:} Model, mevcut gözlemlerine dayanarak bir eylem seçer. Bu eylem, belirli bir hedefe ulaşmak için tasarlanmıştır. Örneğin, bir video oyun karakteri, bir düşmanla savaşmak veya bir engeli aşmak için bir eylem seçebilir.
    \item \textbf{Eylemi Uygulama:} Model seçilen eylemi gerçekleştirir ve çevrede bir değişiklik oluşturur. Örneğin, bir video oyununda karakter hareket eder veya bir silah ateş eder.
    \item \textbf{Geri Bildirim Alınması:} Model, geri bildirim alır. Bu geri bildirim, seçilen eylemin ne kadar başarılı veya başarısız olduğunu belirtir. Örneğin, karakterin canını azaltan bir düşman saldırısı veya bir hedefi vurma başarısı olabilir.
    \item \textbf{Ödül veya Cezalandırma:} Model, aldığı geri bildirime göre bir ödül veya ceza alır. Eylem başarılıysa, genellikle bir ödül alır; başarısızsa bir ceza alır. Örneğin, düşmanı yenmek başarı olarak kabul edilirken, bir engelde takılıp kalmak cezalandırılır.
    \item \textbf{Öğrenme:}Model, aldığı geri bildirime dayanarak öğrenir. Başarılı eylemleri teşvik eden davranışları öğrenirken, başarısız eylemlerden kaçınmayı öğrenir.
    \item \textbf {Yeniden Gözlemleme ve Yeniden Karar Verme:} Model, güncel gözlemlerine dayanarak yeni bir eylem seçer ve süreç tekrarlanır.
\end{enumerate}

    
\section{ML-Agents İçin Gelişmiş Kavramlar}
    Unity ML-Agents'in temel kavramlarına hakim olduktan sonra, gelişmiş teknikler ve kavramlarla daha derinlemesine bir anlayış geliştirebiliriz. Bu bölümde, ML-Agents için önemli olan birkaç gelişmiş kavramı ele alacağız:

    \subsection{Reward Shaping}
        Reward Shaping, ajanın davranışlarını yönlendirmek ve hızlandırmak için ödüllendirme işlemine ek olarak kullanılan bir tekniktir. Bu teknik, ajanın istenen davranışları daha hızlı öğrenmesini sağlayabilir, ancak dikkatlice uygulanmalıdır çünkü yanlış bir şekilde kullanıldığında ajanın performansını olumsuz etkileyebilir.

        \begin{figure}[h]
    \centering
    \includegraphics[width=0.7\textwidth]{odulceza.PNG}
    \caption{Ödül Ceza Sistemi}
    \label{fig:resim14}
    \end{figure}
    

    \subsection{Curriculum Learning}
        Curriculum Learning, ajanın öğrenme sürecini daha etkili hale getirmek için kullanılan bir öğrenme stratejisidir. Bu strateji, ajanın daha basit görevlerle başlamasını ve daha sonra giderek zorlaşan görevlere geçiş yapmasını sağlar. Bu, ajanın daha hızlı ve daha kararlı bir şekilde öğrenmesini sağlayabilir.
        \vspace{0,15cm}
        \footnotesize
        \begin{verbatim}
        private void OnTriggerEnter(Collider other)
        {
            if(other.gameObject.tag=="odun")
            {
                AddReward(10f);
                EndEpisode();
            }
            if(other.gameObject.tag=="wall")
            {
                AddReward(-5f);
                EndEpisode();
            }
        }
\end{verbatim}
        \clearpage

    \subsection{Proximal Policy Optimization (PPO)}
        PPO, ML-Agents'te sıkça kullanılan bir eğitim algoritmasıdır. Policy gradient metotlarının bir türü olan PPO, stabil ve hızlı bir şekilde ajanları eğitmek için tasarlanmıştır. Ayrıca, genellikle büyük ölçekli eğitim setleri üzerinde etkili bir şekilde çalışır.
        \begin{figure}[h]
    \centering
    \includegraphics[width=0.9\textwidth]{yaml.PNG}
    \caption{PPO Örnek Dosyalar}
    \label{fig:resim15}
\end{figure}


\section{Gelişmiş Ortam Tasarımı}
    Gelişmiş bir ML-Agents ortamı tasarlarken dikkate alınması gereken birkaç önemli faktör bulunmaktadır. Bu bölümde, ML-Agents ile gelişmiş ortamların tasarımı ve yönetimi üzerine bazı ipuçlarını ele alacağız:

    \subsection{Kompleks Ortamların Oluşturulması}
        ML-Agents ile kompleks ve etkileşimli ortamlar oluşturmak için Unity'nin özelliklerinden yararlanabiliriz. Bu, ajanların gerçek dünya senaryolarına daha yakın bir şekilde eğitilmesini sağlayabilir.

    \subsection{Çoklu Ajan Ortamları}
        ML-Agents, birden fazla ajanı aynı ortamda eğitmeyi ve simüle etmeyi destekler. Bu, işbirliği ve rekabet temelli senaryolar için idealdir. Çoklu ajan ortamları tasarlarken dikkat edilmesi gereken özel durumlar bulunmaktadır.
        \begin{figure}[h]
    \centering
    \includegraphics[width=0.9\textwidth]{genelgorunum.PNG}
    \caption{Çoklu Eğitim Örneği}
    \label{fig:resim16}
\end{figure}

    
\section{Ajanların Gelişmiş Eğitimi}
    Ajanların gelişmiş eğitimi, ML-Agents kullanıcılarının sık sık karşılaştığı bir konudur. Bu bölümde, ajanların daha hızlı, daha stabil ve daha verimli bir şekilde eğitilmesini sağlamak için kullanılan bazı gelişmiş eğitim tekniklerini ele alacağız:

    \subsection{Eğitim Hiperparametrelerinin Ayarlanması}
        ML-Agents eğitim sürecinde birçok hiperparametre vardır ve bu parametrelerin doğru bir şekilde ayarlanması önemlidir. Örneğin, öğrenme hızı, eğitim devri sayısı, ödül şekillendirme faktörleri gibi parametrelerin dikkatlice ayarlanması gerekebilir.

    \subsection{Eğitim Verimliliğini Artırmak İçin Teknikler}
        Ajanların eğitim verimliliğini artırmak için birçok teknik mevcuttur. Bunlar arasında deneyim tekrarı, örnek çeşitliliği sağlama, eğitim verilerinin dengelemesi gibi teknikler bulunabilir. Bu teknikler, ajanların daha hızlı ve daha stabil bir şekilde öğrenmelerini sağlayabilir.

    \subsection{Hızlandırılmış Eğitim için Paralel Eğitim}
        ML-Agents, paralel eğitim modunu destekler. Bu, birçok ajanın aynı anda eğitilmesini ve eğitim sürecinin hızlandırılmasını sağlar. Paralel eğitim, büyük ölçekli eğitim setleri üzerinde çalışırken özellikle faydalıdır ve eğitim süresini önemli ölçüde azaltabilir.

    
\section{Model Yönetimi ve Hata Ayıklama}
    ML-Agents ile çalışırken, eğitilen modellerin yönetimi ve hata ayıklama süreçleri önemlidir. Bu bölümde, model yönetimi ve hata ayıklama için bazı önemli konuları ele alacağız:

    \subsection{Model Kontrolü ve Yönetimi}
        Eğitilen modellerin kontrolü ve yönetimi, ML-Agents kullanıcılarının sıklıkla karşılaştığı bir konudur. Bu bölümde, eğitim sırasında ve sonrasında modellerin nasıl yönetileceği, kaydedileceği ve yüklenip kullanılacağı gibi konuları ele alacağız.

        \begin{figure}[h]
    \centering
    \includegraphics[width=0.9\textwidth]{onnx.PNG}
    \caption{Model Kontrolü} 
    \label{fig:resim17}
    \end{figure}
    

    \subsection{Eğitim Sırasında Hata Ayıklama}
        ML-Agents ile çalışırken, eğitim sırasında ortaya çıkan hataları tanımlamak ve çözmek önemlidir. Bu bölümde, eğitim sırasında sık karşılaşılan hataları ve bunların nasıl çözülebileceğini ele alacağız. Ayrıca, eğitim sürecini izlemek ve modelin performansını değerlendirmek için kullanılabilecek araçlar hakkında bilgi vereceğiz.


\section{Performans Optimizasyonu}
    ML-Agents ile çalışırken, performans optimizasyonu önemli bir konudur, özellikle büyük ölçekli eğitim setleri veya karmaşık ortamlar kullanırken. Bu bölümde, performansı artırmak için kullanılan bazı teknikleri ele alacağız:

    \subsection{Eğitim Süresini ve Maliyeti Azaltma Yöntemleri}
        Eğitim süresini ve maliyetini azaltmak için kullanılabilecek çeşitli teknikler vardır. Bu teknikler arasında veri paralelleştirmesi, model paralelleştirmesi, dağıtık eğitim ve donanım optimizasyonu gibi yöntemler bulunabilir.

    \subsection{İleri Algoritmaların ve Tekniklerin Kullanımı}
        ML-Agents'in yanı sıra, performansı artırmak için kullanılabilecek daha gelişmiş makine öğrenimi algoritmaları ve teknikler de bulunmaktadır. Bu bölümde, daha gelişmiş algoritmaların ve tekniklerin nasıl kullanılabileceğini ve ML-Agents ile entegrasyonunun nasıl sağlanabileceğini ele alacağız.
\clearpage

\section{Animasyon Oluşturma}

Unity 3D'de animasyon oluşturmak için Animator Controller kullanılır. Animator Controller, birden fazla animasyonu kontrol etmek için kullanılır. Bir nesnenin animasyonlarını oluşturmak için şu adımları izleyebilirsiniz:

\begin{enumerate}
    \item Unity'de bir Animator Controller oluşturun.
    \item Animator Controller'a animasyonlar ekleyin ve geçişler arasında bağlantılar oluşturun.
    \begin{figure}[h]
    \centering
    \includegraphics[width=0.6\textwidth]{animator.PNG}
    \caption{Karakter Animasyonu}
    \label{fig:resim18}
    \end{figure}
    \item Nesneye Animator bileşenini ekleyin ve oluşturduğunuz Animator Controller'ı bağlayın.
\end{enumerate}
\section{Zaman Tutucu Oluşturma}
Unity'de 3D zaman tutucu oluşturmak için bir GameObject içine yerleştirilecek bir script kullanabiliriz.İşte bir örnek:
\begin{verbatim}
public class Timer : Agent
{
    public static float elapsedTime;
    public static float hour=0;
    public static float day=0;
    [SerializeField] private TextMeshProUGUI timerText;
    void Update()
    {
        elapsedTime += Time.deltaTime;
        int minutes =Mathf.FloorToInt(elapsedTime / 60);
        int seconds=Mathf.FloorToInt(elapsedTime % 60);
        // timerText.text = string.Format("{0:00}:{1:00}",minutes,seconds);
        if (hour >= 24)// day passed
        {
            day++;
            hour = 0;
            AddReward(10f);
        }
            hour += Time.deltaTime / 1;
        timerText.text = "Day " + day + "(" + (int)hour + "h)";
    }
}

\end{verbatim}
\begin{figure}[h]
    \centering
    \includegraphics[width=0.6\textwidth]{zamantutucu.PNG}
    \caption{Zaman Tutucu}
    \label{fig:resim19}
    \end{figure}
\section{İterasyon İşlemleri}

İterasyonlar, oyunlarda tekrarlayan işlemleri gerçekleştirmek için kullanılır. Örneğin, bir dizi nesnenin konumunu güncellemek için bir iterasyon kullanabilirsiniz.

\section{Ray Perception Sensor}
Unity ML-Agents, yapay zeka ajanlarının çevrelerini algılamasına yardımcı olmak için Ray Perception Sensor adında bir algı sensörü sunar. Bu sensör, ajanın çevresini ışınlarla algılamasına olanak tanır. Sensör, ajanın önünde bir ışın demeti oluşturur ve bu ışınlar çevredeki nesnelerin konumunu ve türünü belirlemek için kullanılır. Örneğin, bir ajanın önünde duvarlar, engeller veya diğer nesneler varsa, Ray Perception Sensor bu nesnelerin varlığını tespit edebilir.
\begin{figure}[h]
    \centering
    \includegraphics[width=0.9\textwidth]{sensör.PNG}
    \caption{Işın Algılama Sensörü}
    \label{fig:resim20}
    \end{figure}
\subsection{Kullanım}
Ray Perception Sensor, Unity içindeki bir oyun nesnesi olarak kullanılır. Ajanın vücudu üzerinde veya ayrı bir nesne olarak yerleştirilebilir. Daha sonra ML-Agents kodunda bu sensöre erişilebilir ve çevre bilgilerini almak için kullanılabilir.

\subsection{Özellikler}
Ray Perception Sensor'ün bazı temel özellikleri şunlardır:
\begin{itemize}
\item Işın sayısı ve açısı: Algılama için kullanılan ışınların sayısı ve yayılma açısı ayarlanabilir.
\item Algılama katmanları: Algılama için kullanılan farklı katmanlar belirlenebilir, bu sayede sadece belirli nesneler algılanabilir.
\item Algılama mesafesi: Işıların ne kadar uzaklıkta algılama yapacağı belirlenebilir.
\end{itemize}
\begin{figure}[h]
    \centering
    \includegraphics[width=0.9\textwidth]{ray cımp.PNG}
    \caption{Işın Algılama Sensör Özellikleri}
    \label{fig:resim21}
    \end{figure}

\section{Natural Locomotion}
Natural Locomotion, Unity ML-Agents projesindeki bir bileşendir. Bu bileşen, ajanların doğal bir şekilde hareket etmelerine olanak tanır. Örneğin, bir insanın yürümesi gibi, ajanların adımlarını belirli bir hız ve doğrultuda atmalarını sağlar.

\subsection{Kullanım}
Natural Locomotion bileşeni, ajanın hareketini kontrol etmek için kullanılabilir. Ajanın vücudu üzerinde veya ayrı bir nesne olarak yerleştirilebilir. ML-Agents kodunda, ajanın bu bileşen üzerinden kontrol edilmesi sağlanabilir.

\subsection{Özellikler}
Natural Locomotion'ın bazı özellikleri şunlardır:
\begin{itemize}
\item Yürüme hızı ve yönelim: Ajanın yürüme hızı ve yönü ayarlanabilir.
\item Adım uzunluğu ve frekansı: Ajanın adım uzunluğu ve adımlar arası frekans ayarlanabilir.
\item Yerçekimi ve fizik parametreleri: Ajanın fiziksel davranışı üzerindeki etkileri belirlenebilir.
\end{itemize}

\section{Uygulama}

\subsection{Demonstration Recorder Oluşturma}
Unity ML-Agents'ta bir Demonstration Recorder oluşturmak için şu adımları izleyin:
\begin{figure}[h]
    \centering
    \includegraphics[width=0.62\textwidth]{Gösterikaydedici.PNG}
    \caption{Gösteri Kaydedici}
    \label{fig:resim22}
    \end{figure}
\begin{enumerate}
\item \textbf{Recorder Bileşeni Ekle}: Unity sahnenizdeki ajanınıza bir Recorder bileşeni ekleyin.
\item \textbf{Recorder Parametrelerini Ayarlayın}: Recorder bileşenini, gözlemler, eyleml ve ödüller gibi gerekli verileri kaydetmek için yapılandırın.
\item \textbf{Kayıt Başlatma ve Durdurma}: Kullanıcı girişi veya önceden belirlenmiş koşullara bağlı olarak kaydı başlatma ve durdurma işlevselliğini uygulayın.
\end{enumerate}

\subsection{Gösterimleri Kaydetme}
Gösterimleri kaydettikten sonra, bunları uygun formatta kaydedebilirsiniz:

\begin{enumerate}
\item \textbf{Gösterimleri Serileştirme}: Kaydedilen gösterimleri TFRecord veya JSON gibi uygun bir formata serileştirin.
\item \textbf{Diske Kaydetme}: Serileştirilmiş gösterimleri ileride eğitim sırasında kullanmak üzere diske kaydedin.
\end{enumerate}
\begin{figure}[h]
    \centering
    \includegraphics[width=0.62\textwidth]{demokonum.PNG}
    \caption{Disk Kayıt}
    \label{fig:resim23}
    \end{figure}
\section{Kullanım}

\subsection{Gösterimleri Kaydetme}
Gösterimler kaydetmek için şu adımları izleyin:

\begin{enumerate}
\item \textbf{Kaydı Başlatın}: Unity editöründe veya çalışma zamanında gösterim kaydediciyi başlatın.
\item \textbf{Gösterimleri Gerçekleştirme}: Unity ortamında ajanı kullanarak istenen davranışları gösterin.
\item \textbf{Kaydı Durdurun}: Gösterimler tamamlandığında kaydediciyi durdurun.
\end{enumerate}

\subsection{Gösterimlerle Eğitim}
Gösterimleri kaydettikten sonra, ajanlarınızı imitasyonla eğitmek için bunları kullanabilirsiniz. Şu adımları izleyin:

\begin{enumerate}
\item \textbf{Eğitimi Yapılandırma}: Eğitim yapılandırmanızı, kaydedilen gösterimlerle birlikte imitasyon öğrenimini içerecek şekilde ayarlayın.
\item \textbf{Gösterimleri Yükleme}: Eğitim sırasında kaydedilen gösterimleri yükleyin.
\item \textbf{Ajanı Eğitme}: Ajanınızı, hem gösterimler hem de pekiştirmeli öğrenme teknikleri kullanarak eğitin.
\end{enumerate}
\begin{figure}[h!]
\centering
\begin{minipage}[b]{0.4\textwidth}
  \includegraphics[width=\textwidth]{ppo.PNG}
  \caption{Yaml Kodları}
\end{minipage}
\hfill
\begin{minipage}[b]{0.55\textwidth}
  \includegraphics[width=\textwidth]{promptgoruntu.PNG}
  \caption{Anaconda Prompt Çalıştırma Kodu}
\end{minipage}
\end{figure}

\section{Singleton Deseninin Avantajları}
\begin{itemize}
    \item \textbf{Global Erişim}: Singleton, uygulamanın herhangi bir yerinden erişilebilen bir global örnek sağlar.
    \item \textbf{Kontrollü Erişim}: Sadece bir örneğin olmasını garanti eder, böylece sistem kaynaklarının verimli kullanılmasını sağlar.
    \item \textbf{Kolay Yönetim}: Özellikle büyük projelerde, belirli sistemlerin veya yöneticilerin tekil ve merkezi olması, kodun yönetimini ve bakımını kolaylaştırır.
\end{itemize}

\begin{figure}[h]
    \centering
    \includegraphics[width=0.62\textwidth]{hata.PNG}
    \caption{Aldığım Hata }
    \label{fig:resim24}
    \end{figure}
\section{Unity'de Singleton Uygulaması}
Unity'de Singleton deseni uygulamak için aşağıdaki adımları takip edebilirsiniz:

\subsection{Singleton Sınıfının Yazılması}

Öncelikle, Singleton olacak sınıfı oluşturun. İşte basit bir Singleton sınıfının nasıl yazılacağına dair bir örnek:



\subsection{Kodun Açıklaması}
\begin{itemize}
    \item \texttt{private static GameManager \_instance}: Sınıfın tekil örneğini tutan özel bir statik değişken.
    \item \texttt{public static GameManager Instance}: Sınıfın tekil örneğini döndüren ve gerekirse onu oluşturan genel bir statik özellik.
    \item \texttt{void Awake()}: Unity'nin yaşam döngüsünde nesne oluşturulurken çağrılan bir yöntem. Singleton örneğinin var olup olmadığını kontrol eder ve yoksa kendini atar. Varsa, fazladan olan nesneyi yok eder.
\end{itemize}
\begin{figure}[h]
    \centering
    \includegraphics[width=0.62\textwidth]{bestgun.PNG}
    \caption{En iyi gün ve Ödül }
    \label{fig:resim25}
    \end{figure}
\newpage
\section{Algoritma ve Model Seçimi}
\subsection{Gelişmiş Algoritmalar}
Daha hızlı ve etkili öğrenen algoritmalar kullanmak, öğrenme sürecini hızlandırabilir. Örneğin, Proximal Policy Optimization (PPO) ve Advantage Actor-Critic (A2C/A3C) gibi algoritmalar, Q-Learning ve SARSA gibi daha eski yöntemlere göre genellikle daha hızlı öğrenir.
\begin{figure}[h]
    \centering
    \includegraphics[width=0.62\textwidth]{ppo.PNG}
    \caption{Proximal Policy Optimization Algoritması}
    \label{fig:resim26}
    \end{figure}
    \clearpage
\subsection{Model Karmaşıklığı}
Karmaşıklık ve öğrenme hızı arasında denge kurarak, daha basit ve hızlı eğitilebilen modeller kullanmak faydalı olabilir.

\section{Hiperparametre Optimizasyonu}
\subsection{Öğrenme Oranı (Learning Rate)}
Öğrenme oranını dikkatlice ayarlamak, öğrenmenin daha hızlı ve stabil olmasına yardımcı olabilir. Çok yüksek veya çok düşük öğrenme oranları öğrenmeyi yavaşlatabilir.

\subsection{Batch Size}
Uygun batch size seçimi, modelin daha hızlı ve verimli öğrenmesine yardımcı olabilir.

\section{Veri İşleme ve Simülasyonlar}
\subsection{Veri Çeşitliliği ve Miktarı}
Daha fazla ve çeşitli veri sağlamak, modelin genel performansını artırarak öğrenme sürecini hızlandırabilir.
\subsection{Simülasyon Hızı}
Daha hızlı çalışan simülasyon ortamları kullanarak, agentin daha fazla deneyim kazanması sağlanabilir.
\begin{figure}[h]
    \centering
    \includegraphics[width=0.62\textwidth]{egitimsureci.PNG}
    \caption{Eğitim Süreci}
    \label{fig:resim27}
    \end{figure}
\section{Donanım Kullanımı}
\subsection{GPU ve TPU Kullanımı}
Özellikle derin öğrenme tabanlı yöntemlerde, GPU ve TPU gibi güçlü donanımları kullanarak eğitim süresini ciddi ölçüde kısaltmak mümkündür.

\subsection{Dağıtık Hesaplama}
Eğitimi dağıtık olarak yaparak, işlem yükünü birden fazla cihaza yaymak öğrenme süresini hızlandırabilir.

\section{Transfer Öğrenme ve Önyükleme Teknikleri}
\subsection{Transfer Öğrenme}
Daha önce benzer görevlerde eğitilmiş modellerden öğrenerek, yeni görevlerde daha hızlı ve etkili öğrenme sağlanabilir.

\subsection{İnsan Rehberliği}
İnsanların verdiği rehberlik (örneğin, ilkel politikaların öğretilmesi) agentin başlangıçta daha hızlı öğrenmesine yardımcı olabilir.

\section{Ödül Yapısının İyileştirilmesi}
\subsection{Ödül Tasarımı}
İyi tasarlanmış bir ödül yapısı, agentin doğru davranışları daha hızlı öğrenmesine yardımcı olabilir.

\subsection{Shaping Rewards}
Karmaşık görevleri daha küçük ve yönetilebilir parçalara ayırarak ve ara ödüller vererek öğrenme sürecini hızlandırmak mümkündür.
\newpage
\section{Deneyim Tekrarı ve Öncelikli Deneyim Tekrarı}
\subsection{Deneyim Tekrarı (Experience Replay)}
Agentin geçmiş deneyimlerini yeniden kullanarak öğrenmesini sağlamak, öğrenme sürecini hızlandırabilir.
\begin{figure}[h]
    \centering
    \includegraphics[width=0.52\textwidth]{grafik.PNG}
    \caption{Kümülatif Ödül ve Bölüm Uzunluğu}
    \label{fig:resim28}
    \end{figure}
\subsection{Öncelikli Deneyim Tekrarı}
Daha önemli deneyimlerin daha sık tekrarlandığı bir yapı, öğrenmeyi daha verimli hale getirebilir.

\begin{figure}[h]
    \centering
    \includegraphics[width=0.60\textwidth]{enuzunyasadıgısure.PNG}
    \caption{Deneyim Tekrarı}
    \label{fig:resim29}
    \end{figure}
\section{Sonuç}
\begin{itemize}
\item Yapay Zeka Ajanlarının Hayatta Kalma Becerilerinin Gelişimi: Proje, derin güçlendirme öğrenme tekniklerini kullanarak yapay zeka ajanlarının belirli ortamlarda hayatta kalma yeteneklerini nasıl geliştirebileceğimizi gösteriyor. Bu tür teknikler, yapay zekanın gerçek dünya senaryolarında daha etkili bir şekilde işlev görmesine yardımcı olabilir.

\item Doğal Afetler ve Tehlikelerle Başa Çıkma Yeteneği: Proje, yapay zeka ajanlarının doğal afetler, engeller veya diğer tehlikeler gibi zorlu durumlarla nasıl başa çıkabileceğini inceleyerek, bu tür senaryolar için model geliştirmenin önemini vurguluyor. Bu, gerçek dünya uygulamalarında kullanılabilecek önemli bir yetenektir.

\item Derin Güçlendirme Öğrenme Algoritmalarının Uygulaması: Proje, derin sinir ağları ve güçlendirme öğrenme algoritmalarının birleşimini kullanarak yapay zeka modelinin eğitim sürecini nasıl optimize edebileceğimizi gösteriyor. Bu algoritmalar, yapay zeka ajanlarının çevresine uyum sağlamasını ve belirli hedefleri başarmasını sağlamak için optimal eylemleri öğrenmelerine yardımcı olabilir.

\item ML Agents Kullanımı ve Unity Entegrasyonu: Proje, Unity ML Agents aracılığıyla yapay zeka modellerini nasıl eğitebileceğimizi ve gerçek zamanlı simülasyonlarda nasıl kullanabileceğimizi gösteriyor. Bu, özellikle oyun geliştirme alanında yapay zeka teknolojilerinin nasıl entegre edilebileceğini ve geliştirilebileceğini gösteriyor.

\item Ödül Sistemi ve Davranışsal Teşvik: Projenin ödül sistemi, yapay zeka ajanlarının istenilen davranışları öğrenmesini teşvik ediyor. Bu tür teşvik sistemleri, yapay zeka modellerinin belirli görevleri başarıyla tamamlamasını sağlayarak, performanslarını artırabilir ve daha geniş uygulamalar için adapte edilebilir.

\item Sonuç olarak, "AI Learns To Survive" projesi, yapay zeka alanında derin güçlendirme öğrenme tekniklerinin pratik uygulamalarını gösteren ve yapay zeka ajanlarının gerçek dünya senaryolarında nasıl geliştirilebileceğini araştıran önemli bir çalışma olarak öne çıkmaktadır. Bu tür projeler, gelecekte yapay zeka teknolojilerinin günlük hayatta ve endüstride daha geniş bir şekilde nasıl kullanılabileceğini anlamamıza yardımcı olabilir.
\end{itemize}
\newpage
\bibliographystyle{ieeetr}
\bibliography{referance}

\end{document}
